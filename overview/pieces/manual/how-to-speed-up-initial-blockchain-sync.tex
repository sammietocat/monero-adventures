Cite from \url{https://www.monero.how/tutorial-how-to-speed-up-initial-blockchain-sync}
\subsubsection{Overview}
	If you have an SSD drive, there is a way to speed up the initial synchronization of your daemon with the network. This is achieved by downloading a copy of the raw blockchain and importing it. The example below shows the procedure when creating a new wallet, but you can also use the first part of this procedure when preparing to \href{https://www.monero.how/tutorial-restore-monero-command-line-wallet-from-seed.vm}{restore an existing wallet from your seed}. 
	\begin{importantbox}
		It is critical that you only download a trusted version of the raw blockchain from getmonero.org. Do not trust downloads from other sources. 
	\end{importantbox}
	\begin{warningbox}
		If you are booting to a Pendrive or LiveCD, you will not have any hard drive storage available to you, so this tutorial will not work unless you have at least 16GB of RAM available for storage of the blockchain. To clarify, this RAM requirement ONLY applies to Pendrive or LiveCD installations and not to the majority of people with regularly installed operating systems. If you don't have this much RAM in your laptop, \href{https://www.monero.how/tutorial-how-to-create-a-monero-wallet-linux-boot-usb-pendrive}{this tutorial} will show you how to create a Monero wallet on a Live CD booted machine which can talk to a trusted remote daemon. 
	\end{warningbox}
	To begin, open a terminal window (Mac/Linux) or command prompt (Windows) and type the following text that appears in black. Orange text is instructions, and some black text is sample output from the commands you will be running. 
\begin{lstlisting}
# LINUX: Download the Linux 64-bit command line client and extract it
https://downloads.getmonero.org/linux64
tar jxvf linux64

# MAC: Download the Mac command line client and extract it
wget https://downloads.getmonero.org/mac64
tar jxvf mac64

# LINUX/MAC: Download the raw blockchain to get up and running FAST. If you do not trust the source of the download, or want to be safe, skip this step and allow the Monero daemon to synchonize from the network on its own. 
wget https://downloads.getmonero.org/blockchain.raw
./monero-blockchain-import --verify 0 --input-file ./blockchain.raw

# LINUX/MAC: Delete the raw blockchain download as it is no longer needed 
rm -rf ./blockchain.raw

# WINDOWS: Create a new folder with Windows File Explorer, and use your web browser to download the following two files to the new folder
https://downloads.getmonero.org/win64
https://downloads.getmonero.org/blockchain.raw

# WINDOWS: Double click the Monero zip file that has been downloaded to extract it. Then open Command Prompt. Use the 'cd' command to naviate to your new folder
cd monero
monero-blockchain-import.exe --verify 0 --input-file ./blockchain.raw

# WINDOWS: Delete the raw blockchain download as it is no longer needed
del blockchain.raw

# Run the Monero daemon. It will sync with the network and display the message "You are now synchronized with the network. You may now start monero-wallet-cli" when it is completely up to date with the network
# LINUX/MAC:
./monerod
# WINDOWS:
monerod.exe

# The job of the Monero daemon is to monitor the network for new transactions. You now need to open a new terminal window and run the Monero command line wallet, which will communicate with the Monero daemon that you've just stared.
# LINUX/MAC:
./monero-wallet-cli
# WINDOWS:
monero-wallet-cli.exe

# Follow the instructions displayed to create a new wallet. When told the 25 word SEED, write this down on paper and keep it in a very safe place. Even if you forget your passwords, the 25 word SEED can be used to recreate your wallet from any machine and have complete control over your Monero funds. Sample output from the Monero wallet is below:

Specify wallet file name (e.g., MyWallet). If the wallet doesn't exist, it will be created.
Wallet file name (or Ctrl-C to quit): testwallet
No wallet found with that name. Confirm creation of new wallet named: testwallet
(Y/Yes/N/No): Y
Generating new wallet...
Enter a password for your new wallet:  ********
Confirm Password: ********
List of available languages for your wallet's seed:
0 : English
1 : Spanish
2 : German
3 : Italian
4 : Portuguese
5 : Russian
6 : Japanese
Enter the number corresponding to the language of your choice: 0
Generated new wallet: 47CL7FiNtW417VjzWt9Zse8Z8URhaHaA2L9jJq6rrLtDhiYK9PfbCavhhMKws9xEdK HBeGcQtJmPt4uEMivooNztC5UkHLD
View key: 005c98c3db115140289bd0dfad97f910e6eeb5e8e12d02fdd4ab2373fbe9110a
**********************************************************************
Your wallet has been generated!
To start synchronizing with the daemon, use "refresh" command.
Use "help" command to see the list of available commands.
Always use "exit" command when closing monero-wallet-cli to save your
current session's state. Otherwise, you might need to synchronize
your wallet again (your wallet keys are NOT at risk in any case).


PLEASE NOTE: the following 25 words can be used to recover access to your wallet. Please write them down and store them somewhere safe and secure. Please do not store them in your email or on file storage services outside of your immediate control.

locker welders womanly lodge gumball selfish altitude dewdrop
terminal nagged exit acquire hookup ashtray wobbly nineteen
duration duties javelin patio baffles ambush bamboo bite nineteen
**********************************************************************
Starting refresh...
Refresh done, blocks received: 0
Balance: 0.000000000000, unlocked balance: 0.000000000000
Background refresh thread started


# Type "address" to see your public wallet address. You can give this address to anyone, and they will be able to send you Monero. However, NEVER give anyone your 25 word SEED.

[wallet 47CL7F]: address
47CL7FiNtW417VjzWt9Zse8Z8URhaHaA2L9jJq6rrLtDhiYK9PfbCavhhMKws9xEdKHBeGcQtJmPt4uEMivooNz tC5UkHLD
\end{lstlisting}
Congratulations, you have now created a Monero wallet, and can now \href{https://www.monero.how/tutorial-how-to-send-and-receive-monero-command-line}{send and receive Monero}.
