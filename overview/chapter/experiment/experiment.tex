\section{Overview}
To evaluate the effectiveness of the proposed RingCT+ scheme, two criteria of interest are the time complexity of signature generation and latency from network transmission under different circumstances. To compare our RingCT+ and the original RingCT in current Monero, we will estimate the aforementioned criteria under two circumstances, i.e., the same size/number of mixins and the same size of anomynity set. And the detailed methods go as follow. (For convience, the original RingCT in Monero will be simply called RingCT).
\section{Parameters}
	\begin{itemize}
		\item \(l\): the size of a single ring in the mixin matrix (The range of \(l\) to choose is TBD)
		\item \(m\): the number of real input coins to mix (The range of \(l\) to choose is TBD)
		\item \(N\): number of tests on each \( (l,m)\) pair
	\end{itemize}
	Given \(l\) and \(m\), the size of mixins \(S_m\) and the size of the anomynity set \(S_a\) of the RingCT and our RingCT+ can be derived as table
	\begin{table}[!htbp]
		\centering
		\begin{tabular}{ccc}
			\hline
			Schemes & \(S_m\) & \(S_a\) \\
			\hline
			RingCT & \((l-1)\cdot m\) & \(l\) \\
			RingCT+ & \( (l-1)\cdot m\) & \(l^m\) \\
			\hline
		\end{tabular}
	\end{table}
\section{The time complexity of signature generation}
To get rid of the possible influence by other factors, tests to evaluate the time complexity focus on the core function \code{proveRctMG} which is responbile of building the mixin matrix step by step and then calculate the signature based on the resultant mix-in matrix. And for brevity, let's denote the counterpart of function \code{proveRctMG} in the monero project, as \code{proveRctMG2}.
\subsection{Methodology}
	\begin{enumerate}
		\item For each \( (l,m)\) to check, we generate an l-by-m valid public key matrix \(M= [pk_{i,j}]_{l\times m}\) randomly in such a way that all the \(m\) keys in the 1st row are generated with a associated secret key. That's, they are treated as addresses for real input coins.
		\item To evaulate the RingCT signature function, we pass \(M\) and adapt other relevant parameters accordingly to \code{proveRctMG}, and record the time of execution for signing based on these \(M\).
		\item To evaulate the RingCT+ signature function, we pass \(M\) and adapt other relevant parameters accordingly to \code{proveRctMG2}, and record the time of execution for signing based on these \(M\).
		\item Repeat the above process on the same \( (l,m)\) for \(N\) times, and calculate the average time for these \(N\) test cases.
		\item Repeat the above process on other \( (l,m)\) for \(N\) times.
	\end{enumerate}
	By the comparison of the average time for signing under RingCT and RingCT+, put forward the corresponding conclusion. For example, how much overhead stems from the RingCT+.
\section{The latency due to network transmission}
Similarly, tests to estimate latency are conducted subject to the core RPC calls from wallets from daemons.\par
And here parameters are a little different. To achieve the same size of anomynity set (say \(S_a=l^m\)) for \(m\) input coins, the size of a single ring in the RingCT+ would be only \(l\), while it's \(l^m\) for the RingCT. Hence, the dimension of the public key matrix will need to be adjust accordingly as shown in the following methodology part.
\subsection{Methodology}
	\begin{enumerate}
		\item For each \( (l^m,m)\) to check, we generate randomly an l-by-m valid public key matrix \(M_1= [pk_{i,j}]_{l\times m}\) and an \(l^m\)-by-m one \(M_2=[pk_{i,j}]_{l^m\times m}\), in such a way that all the \(m\) keys in the 1st row of both \(M_1\) and \(M_2\) are generated with a associated secret key. That's, they are treated as addresses for real input coins.
		\item Generate signatures for the RingCT and RingCT+ by function \code{proveRctMG} on \(M_1\) and \code{proveRctMG2} on \(M_2\) respectively, and collect the output signature as \(Sig_1\) and \(Sig_2\).
		\item For each \(Sig_i\) (\(i\in {1,2}\)), transfer the signed transaction bound with \(Sig_i\) to a preset standalone daemon, and record the time from the beginning of the transfer and the ending of receving (or parsing?) on the daemon side.
		\item Repeat the above process on the same \( (l^m,m)\) for \(N\) times, and calculate the average time for these \(N\) test cases.
		\item Repeat the above process on other \( (l^m,m)\) for \(N\) times.
	\end{enumerate}
	By the comparison of the average time for transmisson under RingCT and RingCT+, put forward the corresponding conclusion. For example, how much latency has been reduced due to optimised anomynity provided by RingCT+.
